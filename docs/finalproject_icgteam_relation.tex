\documentclass[a4paper,11pt, titlepage]{report}	
\usepackage[T1]{fontenc}			% codifica dei font
\usepackage[utf8]{inputenc}			% lettere accentate da tastiera
\usepackage{gensymb}
\usepackage[italian]{babel}			% lingua del documento
\usepackage{lipsum}					% genera testo fittizio
\usepackage{url}					% per scrivere gli indirizzi Internet
\usepackage{geometry}
\usepackage[nouppercase]{frontespizio}	% suftesi or nouppercase
\usepackage{graphicx}
\usepackage{tabularx}
\usepackage{xcolor, listings}
\usepackage{multirow}
\usepackage[colorlinks]{hyperref}
\usepackage[square,sort,comma,numbers]{natbib}


\geometry{a4paper, top=2cm, bottom=2cm, left=2.5cm, right=2.5cm, heightrounded, bindingoffset=5mm}

\pagestyle{headings}

\linespread{1.25}

\begin{document}

\hypersetup{
	linkcolor=black,
	citecolor=black,
	urlcolor=black
}

\title{
	Interactive Graphics\\
	\large Final Project
	}
\author{Nicola Iommazzo 1693395\\
		Alessandro Basciani 1675251\\
		Matteo Silvestri 1774987\\ }
\date{}
\maketitle

\tableofcontents

%%%%%%%%%%%%%%%%%%%%%%%%%%%%%%%%%%%%%%%%%%%%%%%%%%%%%%%%%%%%%%%%%%%%%%%%%%%%%%%%
%
%	\textbf{GRASSETTO}
%	\texttt{MACCHINA DA SCRIVERE}
%	\textit{CORSIVO}
%	\emph{EMPATIZZATO}
%	'\\' per andare a capo
%	\lipsum per far comparire testo in latino come segnaposto
%
%%%%%%%%%%%

%1 page - M
\chapter{Introduction}
	
	\section{Libreries}

		\subsection{Three.js}

		\subsection{Cannon.js}

%1 page - B
\chapter{Presentation Layers}

%2 page - N
\chapter{Game components}

	\section{Vehicle}
		
	\section{City Buildings}
		
		\subsection{Three.js meshes}
		
		\subsection{Cannon.js bodies}

%1 page - B
\chapter{Hierachical Model}
	
	\section{The NiceDude hierarchical model}

%1 page - B
\chapter{Textures}
	
	\section{City building textures}

	\section{Background scene}

%1 page - M
\chapter{Lights}
	
	\section{General Approach}

%1 page - N
\chapter{User Interaction}

	\section{Vehicle Controller}

%1 page - N
\chapter{Animations}

	\section{NiceDude movements}

%2 page - M
\chapter{Conclusion}
	
	\section{How to test the project}

	\section{Bugs}


\addcontentsline{toc}{part}{\refname}
\begin{thebibliography}{999}
	%\bibitem{codice}
		%cognome n., 
		%\emph{titolo, sottotitolo},
		%casa editrice,
		%anno.
\end{thebibliography}

\end{document}